\documentclass{article}
\usepackage[utf8]{inputenc}
\usepackage{amsmath}
\usepackage{graphicx}
\usepackage{hyperref}
\parindent=0pt
\usepackage[a4paper, total={175mm, 252mm}]{geometry}
\usepackage[section]{placeins}
\usepackage{enumitem}
\usepackage{caption}

\title{Master Thesis Journal}
\author{Möller, Julius}
\date{July 2022}

\begin{document}

\maketitle


\tableofcontents

\pagebreak


\section{INTEGRAL Query}

The first step is to write a class capable of taking retrieving a list of all INTEGRAL pointings observing any given
astronomical object or sky coordinates. Additionally it should be able to filter said list of pointings according to
relevant parameters, such as the start and end times, coordinates, science window version and type, and various other
parameters. The implementation of this class did not require any sophisticated algorithms, and as such will not be
discussed further.


\section{INTEGRAL Pointing Clustering}

The fundamental idea of this project is the following: the INTEGRAL background event rates per detector per energy bin
is assumed to be constant between sufficiently close (in terms of time and orientation) pointings, and can thus be
determined analytically for bunches of several pointings via maximum likelihood calculations. This finally allows us to
apply Bayesian inference fits to determine source parameters. Since these fits are done on each cluster of pointings,
it is pivotal to write an algorithm that clusters a list of INTEGRAL pointings under the following conditions:
\begin{itemize} \label{list:conditions}
    \item The angular distance between pointing orientations should not exceed a predetermined value within each
    cluster.
    \item The angular distance between pointing orientations should be higher than a predetermined value within each
    cluster. Many successive INTEGRAL pointings have nearly identical sky orientations, for which the coded mask
    produces nearly identical patterns; meaning that the above described method of eliminating the background parameter
    in the fits would not work. Hence a minimum angular distance within each cluster is necessary.
    \item The temporal distance between pointings should not exceed a predetermined value within each cluster.
    \item The number number of pointings in each cluster should be within some predetermined range of cluster sizes.
    Depending on the given list of pointings, clusters with fewer pointings than the preferred range may be
    unavoidable, but clusters with more pointings than the preferred range can definitely be avoided. A simple
    mechanism to encourage this is to minimize the total number of clusters, while never exceeding the maximum cluster
    size.
    \item The pointings should be as close as possible in order to justify the assumption that background rates are
    constant. Hence the average effective distance (composed of temporal and angular distance) between pointings should
    be minimized.
\end{itemize}

Clustering points in space together is a common problem in computer science, hence many sophisticated algorithms exist
to do so. However, the above listed conditions are rather unique, and I could not find any established clustering
algorithms that would satisfy them. The common sentiment for clustering algorithms is that there is no such thing as
the perfect clustering algorithm; instead the performance of the algorithm depends on how well it is suited to the type
of distributions found in the data. In our data, the pointings are distributed in three-dimensional space (one time and
two angular coordinates). However, they are not distributed randomly; instead the sky coordinates of INTEGRAL are
adjusted slowly as time progresses. Hence the data are distributed along a line through the time dimension, rather than
randomly distributed in the 3D space. With that in mind, the following clustering algorithm was developed:

\begin{enumerate}
    \item Given some list of pointings looking at an astronomical object, it is quite likely that INTEGRAL spends some
    time looking in the general direction, and then spends a lot of time looking at entirely different sections of the
    sky before returning to the astronomical object. Pointings with large temporal distances will never be clustered
    together, so the run-time of the algorithm can be reduced significantly by splitting the entire query of pointings
    into smaller, disconnected regions, so that the effective distance between any two pointings from different regions
    is larger than the predetermined maximum effective distance. However, in order to split the pointings into
    disconnected regions, the effective distance between any pair of pointings needs to be calculated. Given N
    pointings in our query, this requires computing an N by N matrix, which is computationally very expensive for large
    N. To improve this, the pointings are first split into preliminary regions, such that the temporal distance between
    any two temporally successive pointings in a preliminary region does not exceed the maximum effective distance.
    \item Now that we have reduced our large query into several smaller preliminary regions, we can compute the
    distance matrix for each preliminary region. Hence we have computed the distance matrix for any pair of pointings,
    except that we have only actually computed the distance for any relevant pair of pointings, and setting the value
    for pointings in different preliminary regions to some value higher than the maximum effective distance.
    \item While our preliminary regions are a good start, it is entirely possible that these are composed of several
    actual regions; i.e. that a preliminary regions contains several regions of entirely disconnected pointings, where
    the effective distance between any pair of pointings from different regions is larger than the maximum effective
    distance. Since we have already computed the distance matrix for pointings within preliminary regions, we can use
    the following algorithm to split these into actual regions:
    \begin{enumerate}
        \item \label{itm: region1} Start with some point in the preliminary region, and assign it to a new region while
        removing it from the preliminary region.
        \item \label{itm: region2} Go through the distance matrix, and add any point with a distance smaller than the
        effective distance to the region, and remove said point from the preliminary region.
        \item Repeat the step \ref{itm: region2} for every pointing added to the region in the previous step.
        \item \label{itm: region4} If there are no more pointings to add to the region, repeat from step \ref{itm:
        region1} until the preliminary region is empty.
        \item Repeat steps \ref{itm: region1} to \ref{itm: region4} for all preliminary clusters.
    \end{enumerate}
    \item Now that have split the query into regions, we can cluster each region independently. This is done using the
    following algorithm:
    \begin{enumerate}
        \item First we need create an initial clustering, for which we can take advantage of the fact the the pointings
        are distributed along a temporal line in the 3D parameter space by iterating of the pointings in each regions
        according to the start date:
        \begin{enumerate}
            \item \label{itm: initial clustering 1}Create a cluster from the first un-clustered pointing.
            \item \label{itm: initial clustering 2} Iterate over some predetermined number of temporally successive
            pointings:
            \begin{enumerate}
                \item If the pointing can be added to the cluster without breaking the listed conditions
                \ref{list:conditions}, add it to the cluster and repeat from step \ref{itm: initial clustering 2}. If
                not, continue the iteration.
                \item When the iteration is finished or the cluster has reached is maximum size, continue from
                \ref{itm: initial clustering 1}.
            \end{enumerate}
        \end{enumerate}
        \item While the initial clustering usual produces good results, there is no reason to assume that it is
        anywhere near optimal. For this reason we attempt an improvement on this clustering in the following way:
        \begin{enumerate}
            \item Randomly choose a sub-optimal cluster, weighted by cluster size. This is any cluster with less
            pointings than a predetermined number.
            \item Randomly choose a close sub-optimal cluster, weighted by distance from the first cluster and size. A
            second parameter is used to determine the maximum size of this cluster.
            \item Connect these two sub-optimal clusters through a path of clusters in the following way:
            \begin{enumerate}
                \item \label{itm: clustering path1}Find the pair pointings with the least distance in the two clusters.
                \item Find the the pointings closest to the first of the two above pointings, that are not already part
                of the cluster path. Weigh these using their distance to the pointing, and by the angle between the
                vector connecting the first pointing to the target pointing, and the vector connecting the first
                pointing to the observed pointing. Randomly choose a pointing based on their weights.
                \item If the chosen pointing is in the target cluster, the path is complete. If not, repeat from step
                \ref{itm: clustering path1} using the cluster from the newly selected pointing as the first cluster.
            \end{enumerate}
            \item We will now re-cluster all pointings part of the cluster path:
            \begin{enumerate}
                \item WRITE HERE
            \end{enumerate}
        \end{enumerate}
    \end{enumerate}
\end{enumerate}


\end{document}
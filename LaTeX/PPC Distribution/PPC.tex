\documentclass{article}
\usepackage[utf8]{inputenc}
\usepackage{amsmath}
\usepackage{graphicx}
\usepackage{hyperref}
\parindent=0pt
\usepackage[a4paper, total={175mm, 252mm}]{geometry}
\usepackage[section]{placeins}
\usepackage{enumitem}
\usepackage{caption}

\title{PPC Distributions for  PySPI}
\author{Möller, Julius}
\date{February 2023}

\begin{document}

\maketitle

\section{Introduction}

Given a posterior distribution (from a fit or similar), a standard Posterior Predictive Check (PPC) works in the
following steps:
\begin{enumerate}
    \item Sample from the posterior (i.e. sample parameters for Crab index and intensity)
    \item \label{phys_dataspace} Convert posterior into physical data-space (i.e. flux rate at SPI's position)
    \item Compute instrument response (i.e. SPI's count rate)
    \item \label{rand_sample} Sample instrument data (i.e. measured counts per bin via poisson distribution)
\end{enumerate}
For PySPI, issues arise at step \ref{phys_dataspace}, since the posterior parameter space does not fully describe the
physical system required to predict SPI data. Instead, the entire background is effectively filtered out during the
fitting process using the maximum likelihood value of the background. In order to obtain PPC distributions, we may
recalculate this maximum likelihood background using the posterior samples as source rates and the actual measured
data. This, however, introduces a bias  in our PPC data towards the measured data, which has to taken into account in
step \ref{rand_sample}. The nature of this bias and the resulting distributions will be discussed below.

\section{Assumptions}


\end{document}